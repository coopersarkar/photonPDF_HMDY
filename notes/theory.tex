\section{Theory}
\label{sec:theory}

Two processes contribute to opposite sign, same family, dilepton production at the LHC: 
the Drell-Yan quark-antiquark process and the photon-induced process. Both the contributions can be 
simulated with MadGraph5{\_}aMC@NLO (version 2.4.3) and interfaced to APPLgrid (version 01-04-70) and aMCfast (version 01-03-00). A special release of APPLgrid is used to account for the photon PDF within the proton {\it need references for the programmes}.
Both contributions are generated in the 5-flavour scheme, where all the quarks, except for the \textit{top}
 quark, are treated as massless quarks; all the calculations are performed at fixed-order (FO) without 
parton showers. 

Theoretical predictions for both the one-dimensional $\frac{d\sigma}{dm_{ll}}$ distribution 
(where $m_{ll}$ is the invariant mass of the dilepton pair in the final state) and the double-differential 
distributions $\frac{d^{2}\sigma}{dm_{ll}d|y_{ll}|}$ (where $|y_{ll}|$ is the rapidity of the dilepton pair) 
and $\frac{d^{2}\sigma}{dm_{ll}\Delta\eta_{ll}}$ (where $\Delta\eta_{ll}$ represents the difference in 
pseudorapidity between the two leptons) are generated for both the electron and the muon channels.
 
These predictions are generated using the same selections as in reference~\cite{jhep08-2016-009}
as follows:
\begin{itemize}
\item the invariant mass of the lepton pair is required to be greater than 116 GeV;
\item the absolute value of the pseudorapidity of each lepton is required to be less than 2.5;
\item the transverse momentum ($p_{T}$) of the leading lepton has to be greater than 40 GeV;
\item the $p_{T}$ of the sub-leading lepton has to be greater than 30 GeV.
\end{itemize} 
The binning used is the same as used in reference~\cite{jhep08-2016-009}. For the invariant mass 
distribution, there are 12 bins between 116 GeV and 1.5 TeV with variable bin widths; and for both of the 
 the two-dimensional distributions, there are five different histograms, each one for a different invariant
 mass range: (a) 116 GeV < $m_{ll}$ < 150 GeV; (b) 150 GeV < $m_{ll}$ < 200 GeV; (c) 200 GeV < $m_{ll}$ < 300 GeV; (d) 300 GeV < $m_{ll}$ < 500 GeV; (e) 500 GeV < $m_{ll}$ < 1500 GeV.
 The APPLgrids for the first three $m_{ll}$ intervals are divided into 12 bins with fixed bin 
width between $|y_{ll}^{mim}|$ ($|\Delta\eta_{ll}|$)  = 0.0 (0.0) and $|y_{ll}^{max}|$ ($|\Delta\eta_{ll}|$) = 2.4 (3.0), while the final two $m_{ll}$ intervals are divided into 6 bins with fixed bin width scanning the same $|y_{ll}|$ and $|\Delta\eta_{ll}|$ ranges.

Dynamical renormalization ($\mu_{R}$) and factorization ($\mu_{R}$) scales are used in the calculations 
and both are set to $m_{ll}$. The theoretical calculations were validated by comparing both the NLO QCD + LO EW predictions and the 
LO PI predictions to those computed using the FEWZ 3.1 framework. These calculations are evaluated in the $G_{F}$ electroweak scheme, with the following values for the couplings:
 $\alpha_{S}$ = 0.118; $1/\alpha_{EW}$ = 1/127. The difference between the two predictions is at most 1${\%}$, for both the 1-dimensional and the 2-dimensional distributions.

In order to make a next-to-next-to-leading order (NNLO) fit k-factors ($k_{F}$) are computed matching
 the NLO QCD + LO EW cross sections to higher order (HO) calculations. These are computed using 
FEWZ, with the same input parameters as for the NLO computations. The $k_{F}$ are defined as:
\begin{equation}
k_{F}=\frac{NNLO\  QCD  + NLO\  EW \sigma}{NLO\  QCD + LO\  EW \sigma}
\end{equation}
The MMHT2014NNLO PDF set is used to compute both numerator and denominator.
 The $k_{F}$ are close to the unity and their variation is $\sim 2\%$. {\it provide Table of Final k-factors?}


Discuss theory improvements: addition of the NLO QED+QCD piece 
