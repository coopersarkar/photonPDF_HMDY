\section{Introduction}

Precision phenomenology at the LHC requires theoretical calculations
which include not only QCD corrections, where NNLO is rapidly becoming
the standard, but also electroweak (EW) corrections, which are
particularly significant for observables directly sensitive to the TeV
region.
%
An important ingredient of these electroweak corrections is the photon
parton distribution function (PDF), which must be introduced to absorb
the collinear divergences in initial-state QED emissions.

The first PDF fit to include both QED corrections and a photon PDF was
MRST2004QED~\cite{Martin:2004dh}, where the photon PDF was taken from
a model and tested on HERA data for direct photon production.
%
More recently, the NNPDF2.3QED analysis~\cite{Ball:2013hta} provided a
first model-independent determination of the photon PDF based on
Drell-Yan data from ATLAS and LHCb.
%
The resulting photon PDF is however affected by large uncertainties
due to the limited sensitivity of the data used as input to that fit.
%
The CT group has also recently released a QED fit using a similar
strategy as the MRST2004QED one \cite{Schmidt:2014aba}.

A recent breakthrough concerning the determination of the photon PDF
has been the realisation that it can be expressed in terms of
inclusive lepton-proton deep-inelastic scattering structure functions.
%
The residual uncertainties in the photon PDF resulting from this
strategy, dubbed LUXqed~\cite{Manohar:2016nzj}, are now at the few
percent level, not unlike the quark and gluon PDFs.
%
A related approach by the HKR~\cite{Harland-Lang:2016apc}
group also leads to a similar photon PDF.
%
Also recently, the NNPDF3.0QED set, with improved
QED evolution and state-of-the-art quark and gluon
PDFs, was also made available~\cite{Bertone:2016ume,Ball:2014uwa}.

The aim of this work is to perform a direct determination of the
photon PDF from the recent high-mass Drell-Yan measurements from the
ATLAS experiment at $\sqrt{s}=8$ TeV~\cite{Aad:2016zzw}, and to
compare it with the various existing extractions.
%
As compared to previous measurements of the Drell-Yan processes at
high dilepton invariant masses $m_{ll}$, ATLAS now provides both
single-differential distributions in $m_{ll}$ and double-differential
cross sections in both $m_{ll}$ and $|y_{ll}|$, the rapidity of the
lepton pair, and in $m_{ll}$ and $\Delta\eta_{ll}$, the difference in
pseudo-rapidity between the two leptons.
%
Using the Bayesian reweighting method~\cite{Ball:2011gg,Ball:2010gb}
applied to NNPDF2.3QED, it was shown~\cite{Aad:2016zzw} that these
measurements provided significant information on $x\gamma(x,Q)$, and
thus here we would like to confirm this first studies by means of a
complete PDF fit.

The goal of this study is therefore to investigate further these
constraints from the ATLAS high-mass measurements on the photon PDF,
this time by means of a direct PDF fit performed within the
open-source {\tt xFitter} framework~\cite{Alekhin:2014irh}.
%
State-of-the-art theoretical calculations are employed, in particular
we include NNLO QCD and NLO QED corrections in the PDF evolution and
in the computation of the deep-inelastic-scattering (DIS) structure
functions as implemented in the {\tt APFEL} program and presented from
the first time here.
%
%%%%%%%% 
% Do we want to put conclusion in the introduction?  I would rather
% say:
%This determination of the photon PDF represents an important test of
%the recent developments in theory and data to understand the effects
%of photon PDFs.

%Our results turn out to be in good agreement with the LUXqed and HKR
%calculations, providing further evidence that our understanding
%of the photon PDF has been significantly improved in the last
%few months, both from the point of theory and from
%the point of data.

The outline of this paper is as follows.
%
Sect.~\ref{sec:theory} presents the experimental measurements together
with the theoretical formalism of DIS and Drell-Yan cross-sections
used in this analysis.
%
Then in Sect~\ref{sec:fitsettings} we discuss the settings of the PDF
fit.
% including the parametrization of the photon PDF that will be
% adopted.
%
% Here we also show a sensitivity study which motivates the usefulness
% of the ATLAS measurements for the photon PDF.
%
The fit results are presented in Sect.~\ref{sec:results}, and they are
compared to other existing photon PDF extractions.

%the existing calculations and other
%fits of the photon PDF.
%
Finally, in Sect.~\ref{sec:conclusions} we summarise the results and
discuss future lines of investigation.
%
In addition, appendix~\ref{sec:appendixAPFEL} contains a detailed
description of the implementation and validation of NLO QED
corrections to the DGLAP PDF evolution and DIS structure functions as
implemented in {\tt APFEL}.
