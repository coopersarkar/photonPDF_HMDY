\section{Introduction}

Precision phenomenology at the LHC requires theoretical calculations which
include not only QCD corrections, where NNLO is rapidly becoming
the standard, but also electroweak (EWK) corrections, which become
specially relevant for observables directly sensitive to the TeV region.
%
An important ingredient of these electroweak corrections is
the photon parton distribution function (PDFs), which must
be introduced
to regularise the collinear divergences in initial-state QED emissions.

The first PDF fit to include both QED corrections and a photon PDF
was MRST2004QED, where the photon PDF was taken from a model
and tested on HERA data for direct photon production.
%
More recently, the NNPDF2.3QED set provided a model-independent
determination of the photon PDF based on Drell-Yan data
from ATLAS and LHCb.
%
The resulting photon PDF is however affected by large uncertainties
due to the limited sensitivity of the data used as input to that
fit.
%
The CT group have also released a QED fit using a similar strategy
as the MRST2004QED one.

A recent breakthrough concerning the determination of the
photon PDF has been the realization that it can be expressed
in terms of inclusive lepton-proton deep-inelastic scattering
structure functions.
%
The residual uncertainties in the photon PDF resulting from
this strategy, dubbed LUXqed~\cite{Manohar:2016nzj}, are now at the few percent level,
not unlike the quark and gluon PDFs.
%
A related approach by the HMR group also leads to a similar
photon PDF.

The aim of this work is to perform a direct determination
of the photon PDF from the recent high-mass Drell-Yan measurements
from the ATLAS experiment at $\sqrt{s}=8$ TeV, and compare
with the various existing calculations.
%
As compared to previous measurements of the Drell-Yan processes
at high dilepton invariant masses $m_{ll}$, ATLAS now provides
both single differential distributions in $m_{ll}$ as well
as double-differential cross-sections in both $m_{ll}$
and $|y_{ll}|$, the rapidity of the lepton pair,
and in  $m_{ll}$ and $\Delta\eta_{ll}$, the difference in pseudo-rapidity
between the two leptons.

These more differential distributions provide an additional handle
on the photon PDF, and indeed it was demonstrated in by means
of the Bayesian reweighting method that a significant reduction of
the photon PDF uncertainties of NNPDF2.3QED could be achieved
following the inclusion of the ATLAS data.
%
The goal of this study is to investigate further these constraints
from the ATLAS high-mass measurements on the photon PDF,
this time by means of a direct PDF fit performed within the
open-source {\tt xFitter} framework.
%
State-of-the-art theoretical calculations will be employed,
in particular
we include NNLO QCD and NLO QED corrections, the latter implemented
via the {\tt APFEL} program and presented from the first time
here.
%
Our results turn out to be in good agreement with the LUXqed and HMR
calculations, providing further evidence that our understanding
of the photon PDF has been significantly improved in the last
few months, both from the point of theory and from
the point of data.

The outline of this paper is as follows.
%
First of all in Sect.~\ref{sec:theory} we present the theoretical
calculation of DIS structure functions and of Drell-Yan cross-sections
used in this work.
%
Then in Sect~\ref{sec:fitsettings} we discuss the settings of
the PDF fit, including the parametrization of the photon PDF
that will be adopted.
%
Here we also show a sensitivity study which motivates the
usefulness of the ATLAS measurements for the photon PDF.
%
The results of this work are presented in Sect.~\ref{sec:results},
where we also compare with existing calculations and other
fits of the photon PDF.
%
Finally in Sect.~\ref{sec:conclusions} we conclude
and outline possible future lines of investigation.
%
In addition, appendix~\ref{sec:appendixAPFEL} contains a detailed description of
the implementation and validation of NLO QED corrections
in {\tt APFEL}.
