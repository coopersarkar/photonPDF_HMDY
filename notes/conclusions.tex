\section{Conclusions}

\label{sec:conclusions}

The determination of the photon content of the proton has
attracted a considerable amount of attention recently.
%
In this work, we have presented a determination of the photon PDF from
a PDF fit based on HERA inclusive structure functions and recent ATLAS measurements
of high-mass Drell-Yan cross-sections.
%
We confirm that the high-mass DY data provides significant constraints on the photon PDF
in the region $0.05 \le x \le 0.3$.
%
We find good agreement within uncertainties with two recent calculations of the photon PDF,
LUXqed and HKR for the region of $x$ where the DY data has sensitivity.
%
On the other hand, we also find that a direct determination of the photon PDF
is still far from being competitive with the LUXqed calculation, which uses as input
the high-precision measurements of inclusive structure function of the proton.

The results of this study have been made possible by a number of technical developments
that should be of direct application for future PDF fits accounting for QED corrections,
such as the implementation of $\mathcal{O}\lp alpha^2\rp$ corrections to the DGLAP
evolution and the DIS coefficient functions in {\tt APFEL} or the extension of
{\tt aMCfast} to be able to deal with the photon-initiated calculations provided
by {\tt MadGraph5\_aMC@NLO}.
%
Our results also illustrate the flexibility of the {\tt xFitter} toolbox to extend
its capabilities from the standard quark and gluon PDF fits.


{\bf Acknowledgements}.
%
We thank L. Harland-Lang for providing us a {\tt LHAPDF6} grid
of the HKR photon determination.
%
The work of V.~B., F.~G. and J.~R. has been supported
by the European Research Council Starting Grant ``PDF4BSM".



