\section{Settings}
\label{sec:fitsettings}

In this section we discuss the settings of
the PDF fit, including the parametrization of the photon PDF
that is adopted in the following.
%
The analysis is carried out using the open-source
{\tt xFitter} framework~\cite{Alekhin:2014irh}.
%
The input evolution scale $Q_0$, where the PDFs
are parametrized, is taken to be $Q^2_0 = 7.5~$GeV$^2$.
%
This is also the value $Q^2_0=Q^2_{\rm min}$ used to set the
minimum value of $Q^2$ for those data points that enter the fit.
%
The charm PDF is generated perturbative from quarks and gluons
using the DGLAP equations, exploiting recent developments
in {\tt APFEL} which allow the use of displaced heavy quark
thresholds, so we use $\mu_c=Q_0 > m_c$.

The  expression for the $\chi^2$ used for the fit is the one
defined in Ref.~\cite{Aaron:2009aa}
%
Alternative forms have also been tried,
with no significant difference to our results.
% 
The PDF parametrization input at $Q^2_0$ is determined by the technique of saturation of the $\chi^{2}$, namely one keeps increasing
the number of parameters until the $\chi^{2}$ does not improve further.
%
In this analysis,
the parametrized PDFs are the valence distributions $xu_{v}$ and $xd_{v}$, the gluon distribution $xg$, and the \textit{u}-type and \textit{d}-type sea quarks, $x\bar{U}$, $x\bar{D}$, where $x\bar{U} = x\bar{u}$ and $x\bar{D} = x\bar{d} + x\bar{s}$.
%
In addition, we also parametrize the photon distribution $x\gamma$.
%
The following  functional form is used:
\begin{equation}
  \label{eq:parametrization}
xf(x) = Ax^{B}(1-x)^{C}(1+Dx+Ex^{2})
\end{equation}
where some of the normalization parameters, in particular
$A_{u_{v}}$, $A_{d_{v}}$ and $A_{g}$, are constrained by the valence and momentum
sum rules.
%
The  parameters $B_{\bar{U}}$ and $B_{\bar{D}}$ are set equal to each other, such
the two quark sea distributions share a common small-$x$ behavior.
%
Since the measurements used here are not sensitive to the 
strangeness content of the proton, we fix $x\bar{s} = r_sx\bar{d}$, where $r_s=1.0$ consistent with
the ATLAS 
analysis of inclusive $W$ and $Z$ production~\cite{Aad:2012sb,Aaboud:2016btc}. 
%
A further constraint $A_{\bar{U}} = 0.5 A_{\bar{D}}$ is imposed such that $x\bar{u} \to x\bar{d}$ as $x \to 0$.
The \textit{D} and \textit{E} parameters are added one by one
to the parametrization Eq.~(\ref{eq:parametrization}) until no significant 
improvement in $\chi^{2}$ is found. 

Following this method, the optimal parametrization found for this analysis turns
out to be the following for the quark and gluon PDFs:
\begin{eqnarray}
  \nonumber
  xu_v(x) &&= A_{u_v}x^{B_{u_v}}(1-x)^{C_{u_v}}(1+E_{u_v}x^{2})\, , \\
  \nonumber
xd_v(x) &&= A_{d_v}x^{B_{d_v}}(1-x)^{C_{d_v}}\, , \\
x\bar{U}(x) &&= A_{\bar{U}}x^{B_{\bar{U}}}(1-x)^{C_{\bar{U}}}\, , \\
\nonumber
x\bar{D}(x) &&= A_{\bar{D}}x^{B_{\bar{D}}}(1-x)^{C_{\bar{D}}}\, , \\
\nonumber
\label{eq:param}
xg(x) &&= A_{g}x^{B_{g}}(1-x)^{C_{g}}(1+E_{g}x^{2})\, ,
\end{eqnarray}
while for the photon PDF instead we use
\begin{equation}
x\gamma(x) = A_{\gamma}x^{B_{\gamma}}(1-x)^{C_{\gamma}}(1+D_{\gamma}x+E_{\gamma}x^{2}) \, .
\end{equation}
Note that now the photon PDF also enters the momentum sum rule.

The parametrization of the quark and gluon PDFs Eq.~(\ref{eq:param}) differs from the one used
in the HERAPDF2.0 analysis in various aspects.
%
First of all, we use a higher value of the input evolution scale $Q^2_0$, which is helpful
to stabilize the fit of the photon PDF.
%
Second, the additional negative term in the parametrization of the gluon
is not required here, since we use a more stringent cut $Q_{\rm min}^2$ which removes
some of the low-$x$ data (since they do not provide any information on the photon).
%
Third, the results of the parametrization scan are different due to the presence now
of the ATLAS high-mass Drell-Yan cross-section.

PDF uncertainties are estimated using the Monte Carlo
replica method~\cite{DelDebbio:2004xtd,DelDebbio:2007ee}, cross-checked
with the Hessian method~\cite{Pumplin:2001ct} using $\Delta\chi^2=1$.
%
The former is expected to be more robust than the latter, due to the potential
non-Gaussian nature of the photon PDF uncertainties~\cite{Ball:2013hta}.
%
In addition, we have performed a number of cross-checks to assess the impact
of various model and parametrization uncertainties.
%
For the model uncertainties, we consider variations of the charm mass
between $m_c=1.41$ GeV to 1.53 GeV; of the bottom mass between
$m_c=4.25$ GeV to 4.75 GeV; of the strong
coupling constant $\alpha_s(m_Z)$ between 0.116 to 0.120; as well
as of decreasing the strangeness fraction down to  $r_s=0.75$.


Concerning the parametrization uncertainties, we  quantify the impact of
 increasing the input parametrization scale
 up to $Q_0^2=10$ GeV$^2$, as well as
 of additional parameters in Eq.~(\ref{eq:param}) that make little difference to the $\chi^2$ of the fit, but which can change the shape of the PDFs, in particular the extra negative term for the gluon; $D_{u_v}$, $D_{\bar{u}}$ and $E_{\bar{d}}$.
